\documentclass{article}
\usepackage[a4paper,margin=1in]{geometry}  
\usepackage{graphicx}

\title{AI in Data Documentation: A Helping Hand, Not a Replacement}
\author{Shirin Dehghannezhad}
\date{\today}

\begin{document}

\maketitle

Artificial Intelligence (AI) has undoubtedly changed the way we document data, making the process faster and more structured. I’ve found AI particularly useful for automating repetitive tasks like formatting tables, generating variable descriptions, and ensuring consistency across reports. When used wisely, AI can save hours of manual work and help maintain a professional standard.

That said, I’ve also encountered the downsides. AI-generated documentation often lacks depth—it can misinterpret variable definitions, produce generic explanations, or even miss critical details that require human judgment. It struggles with ambiguity and, if not carefully reviewed, can introduce errors or inconsistencies. 

Through experience, I’ve learned that AI works best as an \textbf{assistant}, not a \textbf{replacement}. It provides a solid starting point, but human oversight is essential for accuracy and clarity. My advice? Use AI to handle the tedious parts, but always double-check its work. The best documentation comes from blending AI’s efficiency with human expertise.

\end{document}
